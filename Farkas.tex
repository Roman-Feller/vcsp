\documentclass[]{article}
\usepackage{amsmath}
\usepackage{amssymb}
\pagenumbering{gobble}

\begin{document}

\section {Extended fields}

Let $F$ be a totally ordered field.
Let $F_\infty = F \cup \{ \bot, \top \}$.
We define operations $+$ and $\cdot$ on $F_\infty$
so that
(for all $a \in F_\infty$ and for all $p \in F, p > 0$)
from top down:
\begin{align*}
	\bot <& \;a\, < \top \\
	\bot + a =& \;\bot = a + \bot \text{\qquad(includes $\bot + \top = \bot$)} \\
	\top + a =& \;\top = a + \top \\
	\bot \cdot\, p =& \;\bot = p \,\cdot \bot \\
	\top \cdot\, p =& \;\top = p \,\cdot \top \\
	\bot \cdot\, 0 =& \;\bot = 0 \,\cdot \bot \text{\qquad(the ``weird'' rule)} \\
	\top \cdot\, 0 =& \;\,0\, = 0 \,\cdot \top
\end{align*}
All other cases of all operations and relations
preserve their behavior from $F$.
We keep the product between negative numbers and
$\{ \bot, \top \}$ undefined.

%\section {Notation}
%
%If $x$ and $y$ are vectors of the same length,
%$x \leqq y$ is elementwise comparison,
%and we write $x \le y$ to denote that
%$x \leqq y$ and $x_i < y_i$ for some index $i$.

\section{Farkas-like conjecture}

Let $A \in F_\infty^{m \times n}$ and $b \in F^m$.
Exactly one of the following is true:
\begin{itemize}
	\item $\exists x \in F^n$ such that
	$0 \leqq x$ and $A x \leqq b$
	\item $\exists y \in F^m$ such that
	$0 \leqq y$ and $(-A^T) y \leqq 0$ and $b^T y < 0$
\end{itemize}

\section{Counterexample}

$$
A =
\begin{pmatrix}
	\bot & \top\\
	\top & \bot
\end{pmatrix}
\qquad \qquad
b =
\begin{pmatrix}
	-1 \\
	-1
\end{pmatrix}
$$
Both are true.

\section{Remark}

For all other versions of Farkas lemma
that I tried to generalize to $F_\infty$
similar counterexample still applies.
However, they might work if it was forbidden
for a row of $A$ to contain both $\bot$ and $\top$
and alike for a column.

The problem is when the conversion to the finite
version requires us to delete both rows and columns,
the rows must be deleted first both in the primar
and in the dual, and so, if deleting a row stops
deleting a column from triggering, it leads to a
different result in the primar than in the dual
because of the matrix transposition.

\end{document}
