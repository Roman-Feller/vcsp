\documentclass[]{article}
\usepackage[portrait, margin=5mm]{geometry}
\usepackage{amsmath}
\usepackage{amssymb}
\usepackage{amsfonts}
\usepackage{listings}
\pagenumbering{gobble}

\begin{document}
	
\lstset{
	basicstyle=\ttfamily\small,
	literate=
	{→}{{$\rightarrow$}}1
	{∀}{{$\forall$}}1
	{∃}{{$\exists$}}1
	{×}{{$\times$}}1
	{σ}{{$\sigma$}}1
	{τ}{{$\tau$}}1
	{≠}{{$\neq$}}1
	{≤}{{$\leq$}}1
	{≥}{{$\geq$}}1
	{↔}{{$\iff$}}1
	{¬}{{$\neg$}}1
	{∧}{{$\wedge$}}1
	{•}{$\bullet$}1
	{·}{$\cdot$}1
	{⬝}{$\cdot$}1
	{ℕ}{{$\mathbb{N}$}}1
	{ₗ}{{$_l$}}1
	{₀}{{$_0$}}1
	{∑}{{$\sum$}}1
	{ᵀ}{{$^\texttt{T}$}}1
	{ᵥ}{{$_v$}}1
}

All variables called $x$ or $y$ are always non-negative (column) vectors.

PRIM:
maximize $c \cdot x$ subject to $A \cdot x \le b$

DUAL:
maximize $(-b) \cdot y$ subject to $(-A^T) \cdot y \le -c$

Theorem (weak duality of linear programming):
If PRIM reaches value $p$ and DUAL reaches value $q$, then
$p \le -q$.

Corollary:
If PRIM is unbounded, DUAL is infeasible.

Corollary:
If DUAL is unbounded, PRIM is infeasible.
\bigskip

Lemma:
If both PRIM and DUAL are feasible, there exists a value $r$ such that
PRIM reaches $-r$ and DUAL reaches $r$.

We are searching for a pair of solutions $x, y$ such that
$-(c \cdot x) = (-b) \cdot y$.

If we are in a ring, it can be simplified to $c \cdot x = b \cdot y$.

Exactly one of the following systems has a solution:
$$
\begin{pmatrix}
	A & 0 \\
	0 & -A^T \\
	-c & b
\end{pmatrix}
\cdot x
\;\le\;
\begin{pmatrix}
	b \\
	-c \\
	0
\end{pmatrix}
$$

Or:
$$
\begin{pmatrix}
	-A^T & 0 & c \\
	0 & A & -b
\end{pmatrix}
\cdot y
\;\le\;
0
$$
$$
\begin{pmatrix}
	b & -c & 0
\end{pmatrix}
\cdot y
\;<\;
0
$$

First we prove that a solution of the first system provides the desired conclusion.

Then we prove that the second system cannot have a solution,
hence the solution provided by the first system always exists.

Let's name parts of the vector $x$ so that the first system of inequalities becomes:
$$
\begin{pmatrix}
	A & 0 \\
	0 & -A^T \\
	-c & b
\end{pmatrix}
\cdot
\begin{pmatrix}
	x \\
	y
\end{pmatrix}
\;\le\;
\begin{pmatrix}
	b \\
	-c \\
	0
\end{pmatrix}
$$

We obtain $A \cdot x \le b$, $(-A^T) \cdot y \le -c$, and
$(-c) \cdot x + b \cdot y \le 0$.
Therefore, $x$ solves PRIM, $y$ solves DUAL, and
$(-c) \cdot x \le -(b \cdot y)$.

If we are in a ring, it means $c \cdot x \ge b \cdot y$ and
with the help of the weak duality, $c \cdot x \le b \cdot y$,
we obtain $c \cdot x = b \cdot y$.
\bigskip

Let's name parts of the vector $y$ so that the second system of inequalities becomes
(the last inequality is sharp):
$$
\begin{pmatrix}
	-A^T & 0 & c \\
	0 & A & -b \\
	b & -c & 0
\end{pmatrix}
\cdot
\begin{pmatrix}
	y \\
	x \\
	z
\end{pmatrix}
\;\le\;
0
$$

We obtain $(-A^T) \cdot y + c \cdot z \le 0$,
$A \cdot x + (-b) \cdot z \le 0$, and
$b \cdot y + (-c) \cdot x < 0$, where
$z$ is a single non-negative number.

If we are in a field and $z$ happens to be strictly positive, we obtain
$(-A^T) \cdot (z^{-1} \bullet y) \le -c$,
hence $(z^{-1} \bullet y)$ solves solves DUAL.

If we are in a field and $z$ happens to be strictly positive, we obtain
$A \cdot (z^{-1} \bullet x) \le b$,
hence $(z^{-1} \bullet x)$ solves solves PRIM.

Using the last inequality, we obtain a contradiction with the weak duality.

Assume now $z = 0$.
We have $(-A^T) \cdot y \le 0$, $A \cdot x \le 0$, and
at least one of $b \cdot y$ or $(-c) \cdot x$ is strictly negative.

In the latter case, $(-c) \cdot x < 0$, that is, $c \cdot x > 0$,
we show that PRIM is unbounded.

For contradiction, let $s$ be a number such that
$c \cdot x < s$ whenever $A \cdot x \le b$.

Since PRIM is feasible, we can take an $x_p$ such that $A \cdot x_p \le b$.

In case $c \cdot x_p > s$, we are done. Assume now $c \cdot x_p \le s$.
Observe that $\frac{s - c \cdot x_p}{c \cdot x}$ is non-negative.

Use $x_p + \frac{s - c \cdot x_p}{c \cdot x} \bullet x$ as
a new solution for PRIM:
$$ A \cdot \left(x_p + \frac{s - c \cdot x_p}{c \cdot x} \bullet x\right)
 = A \cdot x_p + A \cdot \left(\frac{s - c \cdot x_p}{c \cdot x} \bullet x\right)
 = A \cdot x_p + \frac{s - c \cdot x_p}{c \cdot x} \cdot A \cdot x
 \le b + \frac{s - c \cdot x_p}{c \cdot x} \cdot 0
 = b $$
$$ c \cdot \left(x_p + \frac{s - c \cdot x_p}{c \cdot x} \bullet x\right)
 = c \cdot x_p + c \cdot \left(\frac{s - c \cdot x_p}{c \cdot x} \bullet x\right)
 = c \cdot x_p + \frac{s - c \cdot x_p}{c \cdot x} \cdot c \cdot x
 = c \cdot x_p + (s - c \cdot x_p)
 = s $$
 
This contradicts the property of $s$.
Indeed, PRIM is unbounded, so DUAL must be infeasible,
which contradicts our assumption.

In the former case, $b \cdot y < 0$, we show that DUAL is unbounded.

For contradiction, let $s$ be a number such that
$(-b) \cdot y < s$ whenever $(-A^T) \cdot y \le -c$.

Since DUAL is feasible, we can take an $y_q$ such that $(-A^T) \cdot y_q \le -c$.

In case $(-b) \cdot y_q > s$, we are done. Assume now $(-b) \cdot y_q \le s$.
Observe that $\frac{s - ((-b) \cdot y_q)}{b \cdot y}$ is non-positive
and that $A^T \!\cdot y$ is non-negative.

Use $y_q - \frac{s - ((-b) \cdot y_q)}{b \cdot y} \bullet y$ as
a new solution for DUAL:
$$ (-A^T) \cdot \left( y_q - \frac{s - ((-b) \cdot y_q)}{b \cdot y} \bullet y \right)
 = (-A^T) \cdot y_q
 + A^T \cdot \left(\frac{s - ((-b) \cdot y_q)}{b \cdot y} \bullet y \right)
 = (-A^T) \cdot y_q
 + \frac{s - ((-b) \cdot y_q)}{b \cdot y} \cdot A^T \cdot y
 \le -c $$
$$ (-b) \cdot \left( y_q - \frac{s - ((-b) \cdot y_q)}{b \cdot y} \bullet y \right)
 = (-b) \cdot y_q
 + \frac{s - ((-b) \cdot y_q)}{b \cdot y} \cdot b \cdot y
 = (-b) \cdot y_q + (s - ((-b) \cdot y_q))
 = ((-b) \cdot y_q) - ((-b) \cdot y_q) + s
 = s \!\!\!\!\!\!\!\!\!$$
 
This contradicts the property of $s$.
Indeed, DUAL is unbounded, so PRIM must be infeasible,
which contradicts our assumption.


\end{document}
